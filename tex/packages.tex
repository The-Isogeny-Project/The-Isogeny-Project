
\usepackage[dvipsnames,pdftex]{xcolor}
\usepackage{amsmath,amssymb,amsfonts,amsthm}
\usepackage[colorlinks=true,linktocpage]{hyperref}
\usepackage{multirow}
\usepackage{xparse}
\usepackage{siunitx}
\usepackage{enumitem}
\usepackage{relsize}
\usepackage{bm}
\usepackage{tabularx}
\usepackage{arydshln}
\usepackage{stmaryrd}
\usepackage[mathscr]{eucal}
\usepackage{needspace}
\usepackage{pdfpages}
\usepackage{setspace}
\usepackage{orcidlink}
\usepackage{xifthen}
\usepackage{epigraph}
\usepackage[style=long,nolist]{glossaries}
\usepackage[figuresright]{rotating}

% \makeglossaries

\newcounter{ALC@unique}
%\usepackage{algorithmicx}
\usepackage[ruled]{algorithm}
%\usepackage[noend]{algorithmic}
\usepackage[noend]{algpseudocode}


%% to ensure algpseudocode also works as we expect, we renew alot of commands
\newcommand{\REQUIRE}{\Require}
\newcommand{\ENSURE}{\Ensure}
\newcommand{\STATE}{\State}
\newcommand{\WHILE}{\While}
\newcommand{\ENDWHILE}{\EndWhile}
\newcommand{\FOR}{\For}
\newcommand{\ENDFOR}{\EndFor}
\newcommand{\IF}{\If}
\newcommand{\ENDIF}{\EndIf}
\newcommand{\RETURN}{\Return}
\newcommand{\COMMENT}{\Comment}

\usepackage[normalem]{ulem}
%\usepackage{fancyhdr} %clashes with classicalthesis
\usepackage{tikz,tikz-cd}
\usetikzlibrary{positioning}
\usetikzlibrary{calc, shapes, backgrounds}
\usetikzlibrary{arrows, shadows, trees}
\usepackage{tikz-timing}
\usepackage{pifont}

\usepackage{url}
\usepackage{pgfplots}
%\pgfplotsset{compat=1.18}
\pgfplotsset{compat=newest}
\usepackage[stable]{footmisc}
\usepackage{adjustbox}
\usepackage{stackengine}

\usepackage{mathtools}
\providecommand\given{} % so it exists
\newcommand\SetSymbol[1][]{
   \nonscript\,#1\vert \allowbreak \nonscript\,\mathopen{}}
\DeclarePairedDelimiterX\Set[1]{\lbrace}{\rbrace}%
 { \renewcommand\given{\SetSymbol[\delimsize]} #1 }

 
\providecolor{DarkBlue}{rgb}{0,0,.545}
\providecolor{DarkGreen}{rgb}{0,.392,0}
\hypersetup{citecolor=DarkGreen}
\hypersetup{linkcolor=DarkBlue}
\hypersetup{urlcolor=DarkBlue}

\usepackage[nameinlink]{cleveref}
\usepackage{etoolbox}
\usepackage{xfp}  % computations in tables
\usepackage{multirow}
\usepackage[all]{xy}
\usepackage{footmisc}
\usepackage{graphicx}
%\usepackage{cite}
\usepackage{xspace}
\usepackage{multicol}

% for nice tables
\usepackage{array}
\usepackage{booktabs}
\usepackage{makecell}
\usepackage[labelfont=bf]{caption}
\captionsetup[table]{skip=10pt}
\usepackage{subcaption}

\usepackage{subfiles}

\renewcommand{\algorithmicrequire}{\textbf{Input:}}
\renewcommand{\algorithmicensure}{\textbf{Output:}}
\newcommand{\algorithmautorefname}{Algorithm}
\renewcommand{\algorithmiccomment}[1]{\hfill\ $\triangleright$ #1}
%\renewcommand{\algorithmiccomment}[1]{\hfill//\ $\triangleright$ #1}

% TikZ arrowhead/tail styles.
\tikzset{tail reversed/.code={\pgfsetarrowsstart{tikzcd to}}}
\tikzset{2tail/.code={\pgfsetarrowsstart{Implies[reversed]}}}
\tikzset{2tail reversed/.code={\pgfsetarrowsstart{Implies}}}
% TikZ arrow styles.
\tikzset{no body/.style={/tikz/dash pattern=on 0 off 1mm}}


\usepackage[dottedtoc,parts,manychapters,linedheaders]{classicthesis}

% \usepackage{geometry}

% \geometry{
%  a4paper,
%  total={170mm,257mm},
%  left=20mm,
%  top=20mm,
%  }

% %formatting of section titles etc
% \counterwithout{section}{chapter}
% \counterwithout{subsection}{chapter}

\DeclareRobustCommand{\allcaps}[1]{\textls[50]{\scshape #1}}


\makeatletter
\@addtoreset{section}{chapter}
\@addtoreset{definition}{chapter}
\@addtoreset{example}{chapter}
\@addtoreset{problem}{chapter}
\@addtoreset{remark}{chapter}
\@addtoreset{assumption}{chapter}
\makeatother


% arabic for parts, roman for chapters
%\renewcommand{\thepart}{\Roman{part}}
\renewcommand{\thepart}{\arabic{part}}
\renewcommand{\thechapter}{\Roman{chapter}}
\renewcommand{\thesection}{\arabic{section}}
\renewcommand{\thesubsection}{\arabic{section}.\arabic{subsection}}


%In ToC
\renewcommand{\cftchappresnum}{}%
\renewcommand{\cftchapaftersnumb}{\hspace{.8em}\spacedlowsmallcaps}%
\renewcommand{\cftchapfont}{\spacedlowsmallcaps}%
\renewcommand{\cftsecpresnum}{\hspace{.8em}\scshape\MakeTextLowercase}%

%in text
\titleformat{\chapter}[display]%
{\relax}{\raggedleft{\color{CTsemi}\chapterNumber\thechapter} \\ }{0pt}%
{\titlerule\vspace*{.9\baselineskip}\raggedright\spacedallcaps}[\normalsize\vspace*{.8\baselineskip}\titlerule]%


\titleformat{\section}
   {\relax}{\Large\textsc{\S \MakeTextLowercase{\thesection}}}{1em}{\allcaps}

\titleformat{\subsection}
   {\relax}{\textsc{\MakeTextLowercase{\thesubsection}}}{1em}{\spacedlowsmallcaps}

%formatting of header marks
%normally, we want to have it say just Chapter X or so
%if the chapter is unnamed, we want it to say which Part
\def\parttext{Chapter }
\renewcommand{\chaptermark}[1]{\markleft{\spacedlowsmallcaps{
    \ifnum\value{part}>-1
        \parttext \thechapter. \ 
    \else
        {}
    \fi
    #1
    }}}
    
\renewcommand{\sectionmark}[1]{\markright{\spacedlowsmallcaps{
   \ifnum\value{part}>-1
    \ifnum\value{section}>0
        \thechapter.\thesection \
    \else
        \parttext \thechapter.{}
    \fi  
   \fi    
    #1}}}

\makeatletter
\newcounter{tempsaver}
\newcounter{tempsaversec}

\newcommand\partslayout{
    \setcounter{tempsaver}{\value{chapter}}%
    \setcounter{tempsaversec}{\value{section}}%
    \setcounter{chapter}{\value{part}}%
    \setcounter{section}{0}%
    \def\parttext{Part }
}

\newcommand\chapterlayout{
    \setcounter{chapter}{\value{tempsaver}}%
    \setcounter{section}{\value{tempsaversec}}%
    \def\parttext{Chapter }
}
\makeatother
